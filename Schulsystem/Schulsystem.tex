\documentclass[12pt,a4paper]{article}
\usepackage[utf8]{inputenc}
\usepackage[german]{babel}
\usepackage[T1]{fontenc}
\usepackage{amsmath}
\usepackage{amsfonts}
\usepackage{amssymb}
\author{Marco Lents}
\title{Vorschlag für ein besseres Schulsystem}
\begin{document}
\maketitle
\newpage
\tableofcontents
\newpage


\section{Vorwort}
Als Schüler der 12. Klasse des \emph{Hans-Leinberger-Gymnasiums} in Landshut bin ich der Meinung, dass unser Schulsystem sehr ineffizient und verschwenderisch mit den Talenten und Interessen der einzelnen Schüler umgeht. Daher möchte ich hiermit eine bessere Nutzung dieser und anderer Ressourcen vorschlagen.

\section{Zurücklassen des Klassensystems}
Es ist unerlässlich Schüler weder zu unter- noch zu überfordern. Da die Kinder einer Klasse meist nicht das selbe Lerntempo besitzen wäre es sinnvoll dieses Prinzip ganz abzuschaffen. In der heutigen Zeit ist durch digitale Medien möglich auf die individuellen Bedürfnisse und Talente eines jeden einzelnen einzugehen. Lehrer nehmen dann viel mehr die Rolle eines Mentors ein als die des Feindes, wie es doch so oft im Frontalunterricht der Fall ist.

\subsection{Leistungserhebung}
Um einen Anreiz zu schaffen tatsächlich Dinge zu lernen ist es nötig Leistung zu belohnen und das Gegenteil zu bestrafen. Im heutigen Schulsystem übernehmen diese Aufgabe die Zensuren, welche über das Jahr hinweg gesammelt und am Ende in einem Zeugnis zusammengefasst werden. dies ist meiner Meinung nach auch ein System das man am Besten Vergangenheit werden lässt. Zwar ist ein System der Leistungsbewertung unerlässlich, jedoch ist eine Einteilung in 6 bzw. 15 Notenstufen zu undifferenziert. Auch ist die Leistungserhebung zu selten. Anstelle die Leistung eines Schülers kontinuierlich zu bewerten werden in einem Jahr 2-4 große Leistungsnachweise abgelegt zu denen sich einige kleine gesellen. Mein Vorschlag: jeder Schüler kann sich gewisse Lernziele aussuchen. Sein Lernerfolg wird dann an der Bewältigung dieser Lernziele gemessen. In den meisten Fächern ist dies wohl digital möglich, da ich allerdings auch auf Handwerkliche Tätigkeiten in das Schulsystem einbeziehen möchte, kann man wohl doch nicht ganz ohne menschliche Bewertung auskommen. In den Unterrichtsfächern in denen eine digitale Beurteilung möglich ist, ist diese weitaus gerechter als die Bisherige. In allen anderen Fächern sollte die Bewertung jedoch auf wenigstens zwei Fachkundige erledigt werden.
\subsection{Lernziele}
Wie bereits erwähnt können sich die Schüler Lernziele aussuchen. Man darf sie aber nicht wahllos aussuchen lassen, sonst kann es passieren, dass ein Kind, welches gerade aus dem Kindergarten kommt und geradeso die Grundrechenarten beherrscht plötzlich vor dem für es unerreichbaren Ziel der Analysis steht. Daher muss in den ersten Jahren erst einmal der Grundstein für alles was noch kommt gelegt werden. Deshalb kann man anfangs keine Große Auswahl an Lernzielen anbieten. Es ist für jedes Kind in der heutigen Gesellschaft unerlässlich Lesen, Schreiben und Rechnen zu können. Auch sollen die Grundlagen für Handwerkliche Berufe gelehrt werden. Es muss jeder Schüler diese Kurse absolviert haben bis er weitere Lernziele zur Auswahl bekommt. Relativ bald ist es auch nötig Allen Englisch beizubringen.
\end{document}